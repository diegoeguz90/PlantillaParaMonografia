\documentclass[letterpaper]{book}
 
 % https://www.unicauca.edu.co/doctoradoce/publicaciones/Guia_Monografia.pdf 
 
\usepackage[utf8]				{inputenc}
\usepackage[spanish]			{babel}
\usepackage						{amsmath}
\usepackage  					{mathtools}
\usepackage						{amsfonts}
\usepackage						{amssymb}
\usepackage						{graphicx}
\author							{Diego Enrique Guzmán Villamarin}
%% Espaciado
\usepackage						{setspace}
%\onehalfspacing
\setlength{\parindent}			{0pt}
\setlength{\parskip}			{2.0ex plus0.5ex minus0.2ex}

% Otros paquetes
\usepackage  					{float}
\usepackage   					{longtable}
\usepackage   					{lscape}
\usepackage[normalem]			{ulem}
\useunder{\uline}{\ul}{}
% Commands to include C and Matlab code
\usepackage						{listings}
\usepackage   			 		{color}
\usepackage[table,xcdraw]		{xcolor}

% Tabla vertical
\usepackage						{rotating} 
% hipervinculos
\usepackage						{hyperref} 		
% incluye todo en la tabla de contenido excepto el indice
\usepackage[nottoc]				{tocbibind}		
% subfiguras
\usepackage						{subfig}
\usepackage						{multirow}
%\usepackage					{multicolumn}
\usepackage						{bibentry}		
\usepackage						{booktabs}
\usepackage						{epstopdf}
\usepackage[normalem]			{ulem}

% para poner bibliografia
%\nobibliography*

% este paquete para la margen 
\usepackage[top=3cm, bottom=2.5cm, left=2.5cm, right=2.5cm]{geometry}

%\evensidemargin 0.0in
%\oddsidemargin 0.0in

%\setlength{\parskip}{\baselineskip} 	% espacio entre parrafos

% Corrección de palabras desconocidas al separar en silabas:
%\babelhyphenation[english]{every-where}
\hyphenation{de-sa-rro-llo dis-po-si-ti-vo mo-no-cro-ma-ti-co do-cu-men-ta-cion pro-ce-di-mien-to re-a-li-zar ha-bi-li-dad ca-rac-te-ri-zan re-gis-tra-da se-gui-mien-to}
% corrección del tamaño de la letra
\renewcommand{\normalsize}{\fontsize{11}{12}\selectfont}

% ---------------------------------------------------------------------------------
% Para configurar tablas {C=5cm}
% ---------------------------------------------------------------------------------
\usepackage				{array}
\newcolumntype{L}[1]{>{\raggedright\let\newline\\\arraybackslash\hspace{0pt}}m{#1}}
\newcolumntype{C}[1]{>{\centering\let\newline\\\arraybackslash\hspace{0pt}}m{#1}}
\newcolumntype{R}[1]{>{\raggedleft\let\newline\\\arraybackslash\hspace{0pt}}m{#1}}

% ********** Content **********
\begin{document}
	% \renewcommand{\bibname}{\normalsize Referencias} 
	\renewcommand{\listfigurename}{Lista de figuras}
	\renewcommand{\listtablename}{Lista de tablas}
	\renewcommand{\contentsname}{Tabla de contenido}
	\renewcommand{\figurename}{Figura}
	\renewcommand{\tablename}{Tabla} 
	\renewcommand{\baselinestretch}{1.3} % Tamano del interlineado a 1.3
	\renewcommand{\sfdefault}{phv} % Para cambiar el tipo de letra por defecto 
	% Comando para configurar el documento en Arial
	\sffamily % Trabaja con arial en el cuerpo
	\renewcommand{\familydefault}{phv} % Permite trabajar con letra Arial los titulos

	% Aqui se puede incluir lo que vaya antes de la tabla de contenido
	% Portada, contraportada, agradecimientos, etc.	
	\newpage
	\thispagestyle{empty}
	\centering {\uppercase{Titulo del trabajo}}

\centering{\includegraphics[width=0.28\textwidth]{Figures/LogoUni.png}}

\centering {\uppercase{Autor/estudiante}}

\vspace{2cm}

\centering {\uppercase{Para obtar por el titulo de: }}

\vspace{2cm}

\centering {Director(a):} \\
\centering {Nombre completo del tutor} \\
\centering {Titulo del tutor} 

\vspace{2cm}

\centering {Universidad ...} \\
\centering {Facultad ...} \\
\centering {Departamento de ...} \\
\centering {Popayán, \today}
	\newpage
	\thispagestyle{empty}
	\centering {\uppercase{Autor/estudiante}}

\vspace{4cm}

\centering {\uppercase{Titulo del trabajo}}

\vspace{4cm}

Tesis presentada a la Facultad de ... \\
... \\
Universidad ... para la obtención del \\
Título de \\

\vspace{1cm}

Ingeniero en: \\
...

\vspace{1cm}

\centering {Director(a):} \\
\centering {Nombre completo del tutor} \\
\centering {Titulo del tutor}

\vspace{2cm}

Popayán
Año
	\newpage
	\thispagestyle{empty}
	\input{Acceptance.tex}
	\newpage
	\thispagestyle{empty}
	% Poner dedicatoria y en otra pagina agradecimientos
	
	\pagenumbering{Roman} 		% Para comenzar la numeracion de paginas en numeros romanos
	\setcounter{page}{1}		% Inicia la numeración a partir de la siguiente pagina			
	
	% ctrl t - para comentar || ctrl u - para descomentar
	
	\newpage
	\input{Abstract.tex}
	
	\thispagestyle{empty}		% deja esta pagina sin numeración		
	\tableofcontents			% agrega la tabla de contenido
	\thispagestyle{empty}		% deja esta pagina sin numeración		
	\listoffigures 				% indice de figuras
	\thispagestyle{empty}		% deja esta pagina sin numeración		
	\listoftables 				% indice de tablas
		
	% Capitulos, secciones y subsecciones del documento	
	
	\newpage
	\pagenumbering{arabic}		% tipo de numeración
	\setcounter{page}{1}		% Inicia la numeración a partir de la siguiente pagina
	
	% texto numerado con numeros arabigos
	\input{Chapters/Introduction.tex}	
	\input{Chapters/Methods.tex}
	\input{Chapters/Outcomes.tex}
	\input{Chapters/Conclusions.tex}	
	
	% A partir de aqui van las referencias			
	\bibliographystyle{IEEEtran}
	\bibliography{Bibliografia}  
\end{document}